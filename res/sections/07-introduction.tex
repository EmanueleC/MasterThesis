\newpage
\begin{center}
\section*{Acronyms}
\begin{acronym}
\acro{IR}{Information Retrieval}
\acro{DRMM}{Deep Relevance Matching Model}
\acro{BIM}{Binary Independence Model}
\acro{QL}{Query Likelihood}
\acro{MAP}{Mean Average Precision}
\acro{nDCG}{Normalized Discounted Cumulative Gain}
\acro{IDF}{Inverse Document Frequency}
\acro{TREC}{Text REtrieval Conference}
\acro{LTR}{Learning To Rank}
\acro{MLP}{Multi Layer Perceptron}
\acro{W2V}{Word2Vec}
\acro{CBOW}{Continuous Bag Of Words}
\acro{GloVe}{Global Vectors}
\acro{PRF}{Pseudo Relevance Feedback}
\acro{PRIMAD}{Platform, Research goal, Implementation, Method, Actor, Data}
\acro{OOV}{Out Of Vocabulary}
\end{acronym}
\end{center}

\newpage
\chapter{Introduction}

In recent years, there have been dramatic improvements in performance in computer vision, speech recognition, and machine translation tasks.

These improvements were possible thanks to the combination of three factors: (i) advancement in the study of neural network models; (ii) the availability of large datasets and (iii) increased computing power.

In this context, the Information Retrieval (IR) community has just begun to explore neural networks models with the purpose of verifying whether they can be beneficial also in some popular IR tasks (e.g. document ad-hoc retrieval).

Starting from 2013, a new field was born from the intersection of IR and Neural networks: Neural IR, appealing researchers and students.

Although some Neural IR models have indeed produced some improvements over the baselines (w.r.t. document ad-hoc retrieval), the consequence of their application to IR have not yet been completely understood.

There is a strong discussion going on whether IR can benefit from neural networks or not. In fact, there are two main problems with these models: one regards
efficiency (especially when a large collection of documents is considered), the other regards their ability to address the complexity
of IR tasks, i.e. learn patterns in query and document text that indicate \textit{relevance}, even if they use different
vocabulary, and even if the patterns are task-specific or context-specific.

The first problem arises from the long time required by a Neural IR model to compute a similarity score between an (appropriate) learnt
representation of a document and a query. In case of a large corpus, this time becomes prohibitive.

The second problem is especially linked to the difficulty to learn from queries and documents when no large-scale supervised data is available
(unsupervised learning is typically used to learn text representation, while supervised or semi-supervised learning are used to learn ``to match'').

Unfortunately, this is often the case, in fact it is very expensive for a human to label a document ``relevant'' or ``not relevant''
with respect to a certain information need (most of it because relevance is a multifaceted concept).

A couple of strategies have been applied to deal with these problem: to address the first, a re-ranking approach is often used, while for the second, pseudo relevance feedback is used.

Both of them involve the use of a traditional retrieval model in order to obtain supervised data, contributing to the debate on the need of Neural IR.

The rest of this work is organized as follows: in the first three chapters we conducted an analysis of the field(s) of study: IR, artificial neural networks and Neural IR.

Then we reviewed several state-of-the-art works in Neural IR and compared them, pointing out their common characteristics and differences.

In the final chapters we reproduced and replicated the experiment of Guo et al. in \cite{drmm} and critically evaluated it, with a discussion of the results.

The reproduction and replication of the experiment were complex and challenging tasks due to the fact that, even though the source code of DRMM model was publicly available (w/0 preprocessing steps), in the original paper there is no sufficient information to guarantee equal results. The overall work does not limit to the neural network of DRMM model, which is a single component of the system analyzed. The replication tasks involves also preprocessing, word embeddings preparation and the initial retrieval that generates the run to re-rank. Each step has an impact on the performance of DRMM that can increase or decrease the gap between mine results and the originals.

Thus, my contributions are: a critical analysis of an emerging (hybrid) field of study (Neural IR), the reproduction and replication of DRMM, a deep relevance matching model, contributing to the IR community's growing interests in reproducibility and related issues; and making the source code publicly available.
